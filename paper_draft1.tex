\documentclass[12pt]{article}

%\usepackage{comment}
\usepackage{amsfonts}
%\usepackage{latexsym}
\usepackage{amssymb}
\usepackage{amsmath}
\usepackage{graphicx}
\usepackage{float}
\usepackage{epstopdf}
%\usepackage{parskip}

%\setlength{\textwidth}{6.2in}
%\setlength{\oddsidemargin}{0.2in}
%\setlength{\evensidemargin}{0in}
%\setlength{\textheight}{8.9in}
%\setlength{\voffset}{-1.0in}
%\setlength{\parindent}{20pt}
\setlength{\parindent}{1cm}
%\setlength{\mathindent}{1cm}


\newcommand{\beq}{\begin{equation}}
\newcommand{\eeq}{\end{equation}}

\newcommand{\ba}{\begin{array}}
\newcommand{\ea}{\end{array}}

\newcommand{\bea}{\begin{eqnarray}}
\newcommand{\eea}{\end{eqnarray}}

\newcommand{\p}{\partial}
\newcommand{\pp}[2]{\frac{\partial #1}{\partial #2}}
\newcommand{\ppn}[3]{{\partial^{#1} #2 \over \partial #3^{#1}}}
\newcommand{\Pain}{Painlev\'{e} }

\newcommand{\mbf}[1]{\mbox{\boldmath {$#1$}}}
\newcommand{\tx}{\mbox}

\newcommand{\ol}{\overline}
\newcommand{\ft}{\widehat}
\newcommand{\mb}{\mathbb}

%\setlength{\parskip}{2mm}
\renewcommand{\theenumi}{\alph{enumi}}
\renewcommand{\labelenumi}{(\theenumi)}

\begin{document}

%\begin{center}
\title{\bf Adaptive Mesh Refinement for 1-D Hyperbolic PDEs}
\author{ 
Saumya Sinha\footnote{Department of Applied Mathematics, University of Washington, Seattle, Washington
U.S.A.
Email:\texttt{saumya@uw.edu}},
Kenneth J. Roche\footnote{
High Performance Computing Group, 
Pacific Northwest National Laboratory,
1 Battelle Road,
Richland, Washington 
U.S.A. 
Email:\texttt{kenneth.roche@pnnl.gov}} 
}
\maketitle
%\end{center}


{\bf \abstractname{:}} {\small}

\newpage
\tableofcontents
\listoffigures
\listoftables
\newpage

\section{Overview of Paper}
\subsection{Description of AMR (Fig 2.1 from Berger-Leveque)}
\subsection{Wave-propagation for 1D equations -updating grid interfaces (section 4 of paper)}
\subsection{Pseudo-code for AMR}
\subsection{Comment on determining refinement - error estimation}
\subsection{Notes on flagging cells, boundary conditions}

\section{Examples of 1D AMR}
\subsection{scalar advection}
We study here the 1D scalar advection equation
\begin{equation}
q(x,t)_{t} + \bar{u} q(x,t)_{x}=0
\end{equation}
where $\bar{u}$ is a constant representing the velocity of displacement. Let $\bar{u}>0$ for simplicity (right going).

\subsection{variable coefficient (color equation)}
We study here the 1D scalar problem of the form
\begin{equation}
q(x,t)_{t} + u(x) q(x,t)_{x}=0
\end{equation}
\noindent where now the velocity is allowed to vary as a function of its position. This is sometimes called the {\it color} equation.

\subsection{nonlinear scalar (Burgers)}
We study here the 1D systems derived from the conservative, 
\begin{equation}
q(x,t)_{t} + f(q(x,t))_{x}=0
\end{equation}
which can be cast in non-conservative, quasilinear form 
 \begin{equation}
q(x,t)_{t} + f'(q(x,t))q(x,t)_{x}=0 .
\end{equation}
In Burgers inviscid equation, $f(q(x,t))=\frac{1}{2}q^2$.

\subsection{linear system (acoustics)}
We study here the 1D systems of the form
\begin{equation}
q(x,t)_{t} + A q(x,t)_{x}=0
\end{equation}
\noindent and $A$ satisfies hyperbolicity conditions.

\subsection{nonlinear system (shallow water or acoustics with pressure term)}
We use the formulation in chapter 13 of \cite{levFVMHP} to describe the shallow water equations.
So, 

\begin{equation}
\left( \begin{array}{c}
h  \\
h u\end{array} \right)_{t} + 
\left( \begin{array}{c} 
hu \\
hu^2 + \frac{1}{2}gh^2\end{array} \right)_{x} = 0
\end{equation}

\noindent where $h$ is the height (or depth) of the fluid, $u$ is the horizontal component of the velocity, and $hu$ is the discharge -i.e. the flow rate of the fluid past a point. One can assign the state variables $(q^1,q^2)$, and the flux $f$ as follows: 

\begin{equation}
q(x,t)=\left( \begin{array}{c}
h  \\
h u\end{array} \right)_{t} 
=
\begin{equation}
\left( \begin{array}{c}
q^1  \\
q^2\end{array} \right) 
\end{equation}
\noindent and

\begin{equation}
f(q(x,t))=
\left( \begin{array}{c} 
hu \\
hu^2 + \frac{1}{2}gh^2\end{array} \right)
= 
\left( \begin{array}{c} 
q^2 \\
(q^2)^2 / q^1 + \frac{1}{2}g (q^1)^2\end{array} \right)
\end{equation}
\noindent to be consistent with state notation from previous sections of this report.


\section{Results}
\subsection{Numerical experiments for the 1D cases}
\subsection{AMR versus fixed mesh Riemann solvers}
\subsection{Software implementation}

\section{Summary}

%kr -just let the TOC do the outline for us and track our changes at compile time 
%also, I added a couple of bullets since our discussion this afternoon
%\section{Outline}
%\begin{itemize}
%\item Overview of paper
%\begin{itemize}
%\item Description of AMR, Figure 2.1 from Berger-Leveque paper.
%\item Wave propagation form for 1-d equations. Updating on grid interfaces, section 4 of paper
%\item pseudo-code for algorithm, similar to that on page 2310 of paper.
%\item comment on error estimation -determining refinement
%\item notes on: flagging cells, boundary conditions
%\end{itemize}
%\item Examples of 1D AMR
%\begin{itemize}
%\item scalar advection
%\item linear system (e.g. Acoustics)
%\item variable coefficient (e.g. Color equation)
%\item nonlinear scalar (burgers)
%\item nonlinear system (shallow water or acoustics with pressure term)
%\end{itemize}
%\item Results of Numerical Experiments for the 1D cases
%\item Software 
%
%\end{itemize}

\begin{thebibliography}{9}
\bibitem{levFVMHP}  
  R.~Leveque
Finite Volume Methods for Hyperbolic Problems
 Cambridge University Press (2002)  \\

\bibitem{CCS:staff}
 S.~Carter, N.~Rao,
 Private Communication.

\end{thebibliography}

\end{document}